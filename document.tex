\documentclass[12pt]{article}
\pagestyle{headings}
\setlength{\textheight}{8.67in}
\setlength{\textwidth}{5.8in}
\setlength{\topmargin}{-3mm}
\setlength{\headsep}{40pt}
\setlength{\evensidemargin}{-3mm}
\setlength{\parskip}{.05in}
\setlength{\parindent}{3ex}
\renewcommand{\baselinestretch}{1.2}
\usepackage{amsthm}
\usepackage{graphicx}
\setlength{\headsep}{40pt}
\usepackage{graphicx}
\usepackage{color}
\usepackage{array}
\usepackage{amsmath,amssymb}
\usepackage{epstopdf}
\setlength{\topmargin}{-3mm}
\setlength{\leftmargin}{.1in}
\setlength{\footskip}{0.5in}
\pagestyle{myheadings}
\pagenumbering{arabic}
\theoremstyle{definition}
\setlength{\unitlength}{1cm}
\thicklines
\begin{document}
	\title{Linear Transformation in Linear Space}
	\author{Anshada P.M}
	\maketitle
	\section{Linear Transformation}
	Let $ V $ and $W $ be an n dimensional vector space over a field $ \mathbb{F} $. Let $ T :V\rightarrow W $ be a function with $ V $ as its domain and its range contained in $ W $. $$ T(V)\subset W $$ We also assume $ T $ is linear in the sense that $$ T(v_1 + v_2) = T(v_1)+T(v_2) $$ $$ T(\alpha v_1)=\alpha T(v_1)$$ 
	$\forall$ $ v_1,v_2 \in V$ and $\alpha\in\mathbb{F}$.\\

	\begin{picture}(10,3)
	\put(3,2){\circle{4}}
	\put(2.8,2.8){V}
	\put(8,2){\circle{4}}
	\put(7.8,2.8){W}
	\qbezier(3,2)(5.5,4)(7.8,2)
	\put(5.5,2.995){\vector(1,0){0.2}}
	\put(5.5,2.5){$ T $}
	\put(2.7,2){$ v $}
	\put(7.9,2){$ T v $}
	\end{picture}
	\\
	Let $ L(V,W) $ denote the set of linear transformation from $V$ to $W.$ If $ T\in L(V,W),$ $T$ is defined if we prescribe the action of T on a basis of $V$.\\
	\\
    Let $\mathcal{B} = {v_1,v_2,...,v_n}$ be a basis of V.
	Then $v\in V$ given by $v = x_1v_1+x_2v_2+...+x_nv_n$ , $\forall$ $x_i$ $\in \mathbb{F}$
	$$T(v) = T(x_1v_1+x_2v_2+...+x_nv_n)$$
	   $= x_1T(v_1)+x_2T(v_2)+...+x_nT(v_n)$\\
	If we know every $T(v_i)$ we will get T(v).\\              
	\begin{picture}(10,6)
	%\put(2,3.2){\line(1,0){4}}
	\put(2,3.2){\circle{2.5}}
	\put(1.5,4){$ \mathbb{R}^3 $}
	%\put(4,0.2){\line(2,3){2}}
	\put(6,3.2){\circle{2.5}}
	\put(6,4){$ \mathbb{R}^3 $}
	\put(4,0.2){\circle{2.5}}
	\put(2.2,0){$ \mathbb{P}_2(\mathbb{R}) $}
	\put(1.3,3.2){$\phi(u)$}
	\put(5.7,3.4){$\psi(u)$}
	\put(4.,-0.1){u}
	\qbezier(2,3.2)(4,4)(6,3.2)
	\put(4,3.6){\vector(1,0){0.2}}
	\put(3.8,3.9){$ \psi\phi^{-1} $}
	\qbezier(4,0.2)(4,2)(6,3.2)
	\put(4.3,1.6){\vector(1,1){0.2}}
	\put(4.5,1.6){$ \psi $}
	\qbezier(4,0.2)(2,2)(2,3.2)
	\put(2.6,1.7){\vector(-1,1){0.1}}
	\put(2.2,1.6){$ \phi $}
	%\put(4,1.2){\oval(4,2)[r]}
	\end{picture}
	\section{Matrix Representation of Linear Transformation}
	
	
	
\end{document}