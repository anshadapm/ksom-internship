\documentclass[12pt]{article}
\pagestyle{headings}
\setlength{\textheight}{8.67in}
\setlength{\textwidth}{5.8in}
\setlength{\topmargin}{-3mm}
\setlength{\headsep}{40pt}
\setlength{\evensidemargin}{-3mm}
\setlength{\parskip}{.05in}
\setlength{\parindent}{3ex}
\renewcommand{\baselinestretch}{1.2}
\usepackage{amsthm}
\usepackage{graphicx}
\setlength{\headsep}{40pt}
\usepackage{graphicx}
\usepackage{color}
\usepackage{array}
\usepackage{amsmath,amssymb}
\usepackage{epstopdf}
\setlength{\topmargin}{-3mm}
\setlength{\leftmargin}{.1in}
\setlength{\footskip}{0.5in}
\pagestyle{myheadings}
\pagenumbering{arabic}
\theoremstyle{definition}
\setlength{\unitlength}{1cm}
\thicklines
\newtheorem{ex}{Example}
\begin{document}
	\title{\textbf{Linear Transformation on Linear Space} \\
		\large\underline{Internship Report} 
	}
	\author{Anshada P.M}
	\date{July,2019}
	\maketitle
	\pagenumbering{arabic}
	\section{Matrix Representation of Linear Transformation}
	 Let $V$ be an n-dimensional vector space over the field $ \mathbb{F}$ and $W$ an m-dimensional vector space over $\mathbb{F}$. Let $\mathcal{B}$:=$\{u_1,u_2,...,u_n\}$ be an ordered basis for $U$ and $\mathcal{B\prime}$:=$\{v_1,v_2,...,v_n\}$ an ordered basis for $V$. For each linear transformation $T$ from $U$ into $V$, there is an $ m\times n$ matrix $\mathbf{A}$ with entries in $\mathbb{F}$.\\
	 Let $T$ be given by
 	 $$ T(u_{i})=a_{1_{i}}v_1+a_{2_{i}}v_2+...+a_{m_{i}}v_m $$
	  i = (1,2,...,n)\\
	  
	  \section{Similarity Transformation}
	  Let $\mathbf{A}$ and $\mathbf{B}$ be n$\times$n (square) matrices over the field $\mathbb{F}$. We say that $\mathbf{B}$ is similar to $\mathbf{A}$ over $\mathbb{F}$ if there is an invertible n$\times$ n matrix $\mathbf{P}$ over $\mathbb{F}$ such that  $\mathbf{B}$=$\mathbf{P^{-1}}\mathbf{A}\mathbf{P}$.Therefore we will get an equivalence class between $\mathbf{A}$ and $\mathbf{B}$.This relation $\sim$ induces a partition within the matrices.\\
	  Equivalent matrix represents the same linear transformation. The matrices which are not in the same partition cannot involve in the same linear transformation.\\
	  \\
	  $\mathbf{Example:}$Zero matrix is the only element in its partition so as identity matrix.
	  \section{Diagonal Matrix}
	  Does there exist a 'simple' matrix with as many zero entries representing a given linear transformation?\\
	  A simple non-trivial matrix will be diagonal matrix.\\
	  If $f(t)$ is a polynomial in $\mathbb{F}$ and $T$ is a represented by a diagonal matrix ,then is represented by with respect to the same basis
	  
	  \section{Diagonizability}
	  $T$ is diagonizable if there exists an ordered basis for $V$ consisting of eigenvectors of $T$.
	  \subsection{Diagonizable Operators} 
	  Let $T \in L(V)$ be diagonizable.$\exists$ distinct eigenvalues $\lambda_{1},\lambda_{2},...,\lambda_{k}$ each of algebric multiplicity $n_1,n_2,...,n_k$ respectively, and eigen spaces $W_1,W_2,...,W_k$ respectively with dim$(W_i) = n_i.$
	  $$n_1+n_2+...+n_k = n$$
	  $V = W_1\oplus W_2\oplus...\oplus W_k$ ,\space 
	  $(W_i\cap W_j =\{0\}$ for eigen spaces $W_i$ and $W_j$ belonging to the wigenvalues $\lambda_{i}$ and $\lambda_{j}$ whenever $\lambda_{i} \neq \lambda_{j})$
	  That is, $v \in V$ has a unique representation
	 .
	 $$ v = w_1+w_2+...+w_k , \space (w_i\in W_i)$$
	 $$Tw_i = \lambda_{i}w_i$$
	 One can define, $P_i:V\to V$ by $P_iv = w_i$\\
	 Then $P_i$ 's are linear and ${P_i}^2 = P_i$ (Idempotent). Then $P_i$ is called projection on $W_i$ along ${W\prime}_i$ where $${W\prime}_i= W_1\oplus W_2\oplus,,,\oplus W_{i-1}\oplus W_{i+1}\oplus,,,\oplus W_k$$
	 So, \space $V = W_i\oplus {W\prime}_i$\\
	 Now, $v=w_1+w_2+...+w_k = P_1 v+P_2 v+...+P_k v$\\
	 $\implies \boxed{I = P_1+P_2+...+P_k}$ - (1)
	 \\
	 $Tv = Tw_!+Tw_2+...+Tw_k$\\
	 .\hspace{0.5cm}=$\lambda_{1}w_1+\lambda_{2}w_2+...\lambda_{k}w_k$\\
	 .\hspace{0.5cm}=$\lambda_{1}P_1+\lambda_{2}P_2+...\lambda_{k}P_k$\\
	 $\implies$
	 $\boxed{T= \lambda_{1}P_1+\lambda_{2}P_2+...\lambda_{k}P_k}$ - (2)\\
	 \textcircled{1} and \textcircled{2} constitute the celebrated \textbf{Spectral Theorom}.
	 \\
	 
	 \begin{center}
	 ----------x---------------x---------
	 \end{center}
	 Q1. Let $T$ be a linear transformation with respect to the ordered basis $\mathcal{B}:={1,t,t^2}.$
	 $T(1)=3+2t+4t^2$\\
	 $T(t)=2+2t*2$\\
	 $T(t^{2})=4+2t+3t^{2}$\\
	 Analyse this example and verify the spectral theory.\\
	 Ans:\\
	 The matrix representation corresponding to the linear transformation $T$ is 
	 $
	 \begin{bmatrix}
	 3 & 2 & 4\\
	 2 & 0 & 2\\
	 4 & 2 & 3
	 \end{bmatrix}
	 $.\\
	 If $T$ is diagonizable, the det[$T-\lambda I$] = 0 for $\lambda$ is the eigen value.\\
	 $
	 T-\lambda I =
	 \begin{bmatrix}
	 3-\lambda & 2 & 4\\
	 2 & -\lambda & 2\\
	 4 & 2 & 3-\lambda
	 \end{bmatrix}
	 $.\\
	 $
	 det
	 \begin{bmatrix}
	 3-\lambda & 2 & 4\\
	 2 & -\lambda & 2\\
	 4 & 2 & 3-\lambda
	 \end{bmatrix}
	 $ = -($\lambda^3 + 6\lambda^2 + 15\lambda+8$)\\
	 Since det[$T-\lambda I$] = 0, \\
	  $\lambda^3 + 6\lambda^2 + 15\lambda+8$ = 0.\\
	  $(\lambda-8){(\lambda+1)}^2 = 0$\\
	  So, the eigen values are 8,-1,-1.\\
	  \\
	  When $\lambda$ = 8,\\
	 $
	 [T-8I] = 
	 \begin{bmatrix}
	 -5 & 2 & 4\\
	 2 & -8 & 2\\
	 4 & 2 & -5
	 \end{bmatrix}
	 $ \\
	 $
	 \begin{bmatrix}
	 -5 & 2 & 4\\
	 2 & -8 & 2\\
	 4 & 2 & -5
	 \end{bmatrix}
	 \begin{bmatrix}
	 x_1\\
	 x_2\\
	 x_3
	 \end{bmatrix}
	 =
	 \begin{bmatrix}
	 0\\
	 0\\
	 0
	 \end{bmatrix}
	 $ \\
	 $\implies
	 2x_1+-8x_2+2x_3 = 0.\\
	 4x_1+-2x_2+-5x_3 = 0.\\
	 \implies x_1 = x_3,x_1 = 2x_2$.\\
	 So.
	 $
	 \begin{bmatrix}
	 x_1\\
	 x_2\\
	 x_3
	 \end{bmatrix}
	 =
	 \begin{bmatrix}
	 2x_2\\
	 x_2\\
	 2x_2
	 \end{bmatrix}
	 $\\
	 When $x_2$ = 1,
	 $
	 \begin{bmatrix}
	 2x_2\\
	 x_2\\
	 2x_2
	 \end{bmatrix}
	 =
	 \begin{bmatrix}
	 2\\
	 1\\
	 2
	 \end{bmatrix}
	 $\\
	 $W_1 = Nullspace(T-8I)$.\\
	 \\
	 When $\lambda$ = -1,\\
	 $
	 [T-(-1)I] = 
	 \begin{bmatrix}
	 4 & 2 & 4\\
	 2 & 1 & 2\\
	 4 & 2 & 4
	 \end{bmatrix}
	 $ \\
	 $
	 \begin{bmatrix}
	 4 & 2 & 4\\
	 2 & 1 & 2\\
	 4 & 2 & 4
	 \end{bmatrix}
	 \begin{bmatrix}
	 x_1\\
	 x_2\\
	 x_3
	 \end{bmatrix}
	 =
	 \begin{bmatrix}
	 0\\
	 0\\
	 0
	 \end{bmatrix}
	 $ \\
	 $\implies
	 2x_1+1x_2+2x_3 = 0.\\
	 x_1 =0,x_2=-2,x_3=1.\\
	 x_2=0,x_1=1.x_3=-1.$
	 So,$
	 \begin{bmatrix}
	 x_1\\
	 x_2\\
	 x_3
	 \end{bmatrix}
	 =
	 \begin{bmatrix}
	 0\\
	 -2\\
	 1
	 \end{bmatrix}
	 and
	 \begin{bmatrix}
	 1\\
	 0\\
	 -1
	 \end{bmatrix}
	 $\\
	  $(x_1,x_2,x_3)\in {\mathbb{R}}^3$ then, $p(t)= x_1+x_2t+x_3t^2\\
	  (x_1,x_2,x_3)=y_1(2,1,2)+y_2(0,-2,1)+y_3(1,0,-1)\\
	  \hspace*{1.87cm}=((2y_1+y_3) , (y_1-2y_2) ,(2y_1+y_2-y_3))\\
	  2y_1+y_3 = x_1,\\y_1-2y_2=x_2,\\2y_1+y_2-y_3=x_3\\
	  \implies y_1 = \frac{1}{9}(2x_1+1x_2+2x_3)\\
	  \hspace*{0.9cm}y_2 = \frac{1}{9}(1x_1+-4x_2+1x_3)\\
	  \hspace*{0.9cm}y_3 = \frac{1}{9}(5x_1+-2x_2+-4x_3)	 \\
	  \therefore(x_1,x_2,x_3)=[\frac{1}{9}(2x_1+1x_2+2x_3)](2,1,2)+\frac{1}{9}(x_1+-4x_2+x_3)](0,-2,1)+[\frac{1}{9}(5x_1+-2x_2+-4x_3)](1,0,-1).\\
	  \\
	  P_1(x_1,x_2,x_3)=[\frac{1}{9}(2x_1+1x_2+2x_3)](2,1,2)]\\
	  .\hspace{2.2cm} = [\frac{1}{9}(4x_1+2x_2+4x_3),\frac{1}{9}(2x_1+1x_2+2x_3),\frac{1}{9}(4x_1+2x_2+4x_3)]\\$
	  So 
	  $
	  [P_1] = 
	  \begin{bmatrix}
	  \frac{4}{9} & \frac{2}{9} & \frac{4}{9}\\
	  \frac{2}{9} & \frac{1}{9} & \frac{2}{9}\\
	  \frac{4}{9} & \frac{2}{9} & \frac{4}{9}
	  \end{bmatrix}
	  \\
	  P_2(x_1,x_2,x_3)=\frac{1}{9}(x_1+-4x_2+x_3)](0,-2,1)+[\frac{1}{9}(5x_1+-2x_2+-4x_3)](1,0,-1)\\
	  .\hspace{2.2cm} = [\frac{1}{9}(5x_1+-2x_2+-4x_3),\frac{1}{9}(-2x_1+8x_2+-2x_3),\frac{1}{9}(-4x_1+-2x_2+5x_3)]\\$
	  So 
	  $
	  [P_2] = 
	  \begin{bmatrix}
	  \frac{5}{9} & \frac{-2}{9} & \frac{-4}{9}\\
	  \frac{-2}{9} & \frac{8}{9} & \frac{-2}{9}\\
	  \frac{-4}{9} & \frac{-2}{9} & \frac{5}{9}
	  \end{bmatrix}
	  \\
	  \\
	  $
	  $[P_1]+[P_2]=\begin{bmatrix}
	  \frac{4}{9} & \frac{2}{9} & \frac{4}{9}\\
	  \frac{2}{9} & \frac{1}{9} & \frac{2}{9}\\
	  \frac{4}{9} & \frac{2}{9} & \frac{4}{9}
	  \end{bmatrix}
	  +
	  \begin{bmatrix}
	  \frac{5}{9} & \frac{-2}{9} & \frac{-4}{9}\\
	  \frac{-2}{9} & \frac{8}{9} & \frac{-2}{9}\\
	  \frac{-4}{9} & \frac{-2}{9} & \frac{5}{9}
	  \end{bmatrix}
	  =
	  \begin{bmatrix}
	  1 & 0 & 0\\
	  0 & 1 & 0\\
	  0 & 0 & 1
	  \end{bmatrix}
	  =
	  I\\
	  \lambda_{1}[P_1]+\lambda_{2}[P_2]=
	  8
	  \begin{bmatrix}
	  \frac{4}{9} & \frac{2}{9} & \frac{4}{9}\\
	  \frac{2}{9} & \frac{1}{9} & \frac{2}{9}\\
	  \frac{4}{9} & \frac{2}{9} & \frac{4}{9}
	  \end{bmatrix}
	  +
	  -1
	  \begin{bmatrix}
	  \frac{5}{9} & \frac{-2}{9} & \frac{-4}{9}\\
	  \frac{-2}{9} & \frac{8}{9} & \frac{-2}{9}\\
	  \frac{-4}{9} & \frac{-2}{9} & \frac{5}{9}
	  \end{bmatrix}
	  $\\
	  \hspace*{2.8cm}=
	  $
	  \begin{bmatrix}
	  3 & 2 & 4\\
	  2 & 0 & 2\\
	  4 & 2 & 3
	  \end{bmatrix} = [T]
	  \\
	  $
	  Thus the properties of spectral theory have been verified with the help of the example.
	  Moreover,the linear transformation $T$ can be also expressed in terms of a diagonal matrix with the ordered basis $\mathcal{B\prime}=\{(2+t+2t^2),(-2t+t^2),(1-t^2)\}$.
	  $
	  {[T]}_{\{(2+t+2t^2),(-2t+t^2)(1-t^2)\}} = 
	  \begin{bmatrix}
	  8 & 0 & 0\\
	  0 & -1 & 0\\
	  0 & 0 & -1
	  \end{bmatrix}
	  $.
	  \section{Linear Transformation}
	  Let $ V $ and $W $ be an n dimensional vector space over a field $ \mathbb{F} $. Let $ T :V\rightarrow W $ be a function with $ V $ as its domain and its range contained in $ W $. $$ T(V)\subset W $$ We also assume $ T $ is linear in the sense that $$ T(v_1 + v_2) = T(v_1)+T(v_2) $$ $$ T(\alpha v_1)=\alpha T(v_1)$$ 
	  $\forall$ $ v_1,v_2 \in V$ and $\alpha\in\mathbb{F}$.\\
	  
	  \begin{picture}(10,3)
	  \put(3,2){\circle{4}}
	  \put(2.8,2.8){V}
	  \put(8,2){\circle{4}}
	  \put(7.8,2.8){W}
	  \qbezier(3,2)(5.5,4)(7.8,2)
	  \put(5.5,2.995){\vector(1,0){0.2}}
	  \put(5.5,2.5){$ T $}
	  \put(2.7,2){$ v $}
	  \put(7.9,2){$ T v $}
	  \end{picture}
	  \\
	  Let $ L(V,W) $ denote the set of linear transformation from $V$ to $W.$ If $ T\in L(V,W),$ $T$ is defined if we prescribe the action of T on a basis of $V$.\\
	  \\
	  Let $\mathcal{B} = {v_1,v_2,...,v_n}$ be a basis of V.
	  Then $v\in V$ given by $v = x_1v_1+x_2v_2+...+x_nv_n$ , $\forall$ $x_i$ $\in \mathbb{F}$
	  $$T(v) = T(x_1v_1+x_2v_2+...+x_nv_n)$$
	  $= x_1T(v_1)+x_2T(v_2)+...+x_nT(v_n)$\\
	  If we know every $T(v_i)$ we will get T(v).\\              
	  \begin{picture}(10,6)
	  %\put(2,3.2){\line(1,0){4}}
	  \put(2,3.2){\circle{2.5}}
	  \put(1.5,4){$ \mathbb{R}^3 $}
	  %\put(4,0.2){\line(2,3){2}}
	  \put(6,3.2){\circle{2.5}}
	  \put(6,4){$ \mathbb{R}^3 $}
	  \put(4,0.2){\circle{2.5}}
	  \put(2.2,0){$ \mathbb{P}_2(\mathbb{R}) $}
	  \put(1.3,3.2){$\phi(u)$}
	  \put(5.7,3.4){$\psi(u)$}
	  \put(4.,-0.1){u}
	  \qbezier(2,3.2)(4,4)(6,3.2)
	  \put(4,3.6){\vector(1,0){0.2}}
	  \put(3.8,3.9){$ \psi\phi^{-1} $}
	  \qbezier(4,0.2)(4,2)(6,3.2)
	  \put(4.3,1.6){\vector(1,1){0.2}}
	  \put(4.5,1.6){$ \psi $}
	  \qbezier(4,0.2)(2,2)(2,3.2)
	  \put(2.6,1.7){\vector(-1,1){0.1}}
	  \put(2.2,1.6){$ \phi $}
	  %\put(4,1.2){\oval(4,2)[r]}
	  \end{picture}
	  Example:1
	  $T(1)=5+1t+3t^2$\\
	  $T(t)=-6+4t-6t^2$\\
	  $T(t^{2})=-6+2t+-4t^{2}$\\
	  Ans:\\
	  The matrix representation corresponding to the linear transformation $T$ is 
	  $
	  \begin{bmatrix}
	  5 & -6 & -6\\
	  -1 & 4 & 2\\
	  3 & -6 & 4
	  \end{bmatrix}
	  $.\\
	  If $T$ is diagonizable, the det[$T-\lambda I$] = 0 for $\lambda$ is the eigen value.\\
	  $
	  T-\lambda I =
	  \begin{bmatrix}
	  5-\lambda & -6 & -6\\
	  -1 & 4-\lambda & 2\\
	  3 & -6 & 4-\lambda
	  \end{bmatrix}
	  $.\\
	  $
	  det
	  \begin{bmatrix}
	  5 & -6 & -6\\
	  -1 & 4 & 2\\
	  3 & -6 & 4
	  \end{bmatrix}
	  $ = -($\lambda^3 - 3\lambda^2 + 6\lambda-4$)\\
	  Since det[$T-\lambda I$] = 0, \\
	  $\lambda^3 + 3\lambda^2 -6\lambda-4$ = 0.\\
	  $(\lambda-1){(\lambda-2)}^2 = 0$\\
	  So, the eigen values are 1,2,2.\\
	  \\
	  When $\lambda$ = 1,\\
	  $
	  [T-1I] = 
	  \begin{bmatrix}
	  4 & -6 & -6\\
	  -1 & 3 & 2\\
	  3 & -6 & 3
	  \end{bmatrix}
	  $ \\
	  $
	  \begin{bmatrix}
	  4 & -6 & -6\\
	  -1 & 3 & 2\\
	  3 & -6 & 3
	  \end{bmatrix}
	  \begin{bmatrix}
	  x_1\\
	  x_2\\
	  x_3
	  \end{bmatrix}
	  =
	  \begin{bmatrix}
	  0\\
	  0\\
	  0
	  \end{bmatrix}
	  $ \\
	  $\implies
	  4x_1-6x_2-6x_3 = 0.\\
	  -x_1+3x_2+2x_3 = 0.\\
	  \implies x_1 = x_3,x_1 = -3x_2$.\\
	  So.
	  $
	  \begin{bmatrix}
	  x_1\\
	  x_2\\
	  x_3
	  \end{bmatrix}
	  =
	  \begin{bmatrix}
	  -3x_2\\
	  x_2\\
	  -3x_2
	  \end{bmatrix}
	  $\\
	  When $x_2$ = -1,
	  $
	  \begin{bmatrix}
	  -3x_2\\
	  x_2\\
	  -3x_2
	  \end{bmatrix}
	  =
	  \begin{bmatrix}
	  3\\
	  -1\\
	  3
	  \end{bmatrix}
	  $\\
	  $W_1 = Nullspace(T-I)$.\\
	  \\
	  When $\lambda$ = 2,\\
	  $
	  [T-2I] = 
	  \begin{bmatrix}
	  3 & -6 & -6\\
	  -1 & 2 & 2\\
	  3 & -6 & 2
	  \end{bmatrix}
	  $ \\
	  $
	  \begin{bmatrix}
	  3 & -6 & -6\\
	  -1 & 2 & 2\\
	  3 & -6 & -6
	  \end{bmatrix}
	  \begin{bmatrix}
	  x_1\\
	  x_2\\
	  x_3
	  \end{bmatrix}
	  =
	  \begin{bmatrix}
	  0\\
	  0\\
	  0
	  \end{bmatrix}
	  $ \\
	  $\implies
	  -x_1+2x_2+2x_3 = 0.\\
	  \implies x_1 = 0, x_2=1,x_3=-1.\\
	  x_1 = 2, x_2=0,x_3=-1$.\\
	  $W_2 = Nullspace(T-2I)$.\\
	  $\{(0,1,-1),(2,0,1)\}$ spans $W_2$.\\
	  The linear transformation $T$ can be also expressed in terms of a diagonal matrix with the ordered basis $\mathcal{B\prime}=\{(1+2t+2t^2),(t-t^2),(2+t^2)\}.\\
	  p_1 = 1+2t+2t^2\\
	  p_2 = t-t^2\\
	  p_3 = 2+t^2\\$
	  $p_1,p_2$ and $p_3$ are the eigen vectors of T.\\
	  $
	  Tp_1 = 1p_1+0p_2+0p_3\\
	  Tp_1 = 0p_1+20p_2+0p_3\\
	  Tp_1 = 0p_1+0p_2+2p_3\\
	  {[T]}_{\{p_1,p_2,p_3\}}=
	  \begin{bmatrix}
	  1 & 0 & 0\\
	  0 & 2 & 0\\
	  0 & 0 & 2
	  \end{bmatrix}
	  $
\end{document}	
