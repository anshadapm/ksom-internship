\documentclass[12pt]{article}
\pagestyle{headings}
\setlength{\textheight}{8.67in}
\setlength{\textwidth}{5.8in}
\setlength{\topmargin}{-3mm}
\setlength{\headsep}{40pt}
\setlength{\evensidemargin}{-3mm}
\setlength{\parskip}{.05in}
\setlength{\parindent}{3ex}
\renewcommand{\baselinestretch}{1.2}
\usepackage{amsthm}
\usepackage{graphicx}
\setlength{\headsep}{40pt}
\usepackage{graphicx}
\usepackage{color}
\usepackage{array}
\usepackage{amsmath,amssymb}
\usepackage{epstopdf}
\setlength{\topmargin}{-3mm}
\setlength{\leftmargin}{.1in}
\setlength{\footskip}{0.5in}
\pagestyle{myheadings}
\pagenumbering{arabic}
\theoremstyle{definition}
\setlength{\unitlength}{1cm}
\thicklines
\begin{document}
	\title{Linear Transformation in Linear Space}
	\author{Anshada P.M}
	\maketitle
	\section{Linear Transformation}
	Let $ V $ and $W $ be an n dimensional vector space over a field $ \mathbb{F} $. Let $ T :V\rightarrow W $ be a function with $ V $ as its domain and its range contained in $ W $. $$ T(V)\subset W $$ We also assume $ T $ is linear in the sense that $$ T(v_1 + v_2) = T(v_1)+T(v_2) $$ $$ T(\alpha v_1)=\alpha T(v_1)$$ 
	$\forall$ $ v_1,v_2 \in V$ and $\alpha\in\mathbb{F}$.\\

	\begin{picture}(10,3)
	\put(3,2){\circle{4}}
	\put(2.8,2.8){V}
	\put(8,2){\circle{4}}
	\put(7.8,2.8){W}
	\qbezier(3,2)(5.5,4)(7.8,2)
	\put(5.5,2.995){\vector(1,0){0.2}}
	\put(5.5,2.5){$ T $}
	\put(2.7,2){$ v $}
	\put(7.9,2){$ T v $}
	\end{picture}
	\\
	Let $ L(V,W) $ denote the set of linear transformation from $V$ to $W.$ If $ T\in L(V,W),$ $T$ is defined if we prescribe the action of T on a basis of $V$.
	\\
    Let $\mathcal{B} = \{v_1,v_2,...,v_n\}$ be a basis of $V$.
	Then $v\in V$ given by $v = x_1v_1+x_2v_2+...+x_nv_n$ , $\forall$ $x_i$ $\in \mathbb{F}$
	$$T(v) = T(x_1v_1+x_2v_2+...+x_nv_n)$$
	$$    = x_1T(v_1)+x_2T(v_2)+...+x_nT(v_n)$$
	If we know each $T(v_i)$ we will get $T(v)$.\\              
	\begin{picture}(10,6)
	%\put(2,3.2){\line(1,0){4}}
	\put(2,3.2){\circle{2.5}}
	\put(1.5,4){$ \mathbb{R}^3 $}
	%\put(4,0.2){\line(2,3){2}}
	\put(6,3.2){\circle{2.5}}
	\put(6,4){$ \mathbb{R}^3 $}
	\put(4,0.2){\circle{2.5}}
	\put(2.2,0){$ \mathbb{P}_2(\mathbb{R}) $}
	\put(1.3,3.2){$\phi(u)$}
	\put(5.7,3.4){$\psi(u)$}
	\put(4.,-0.1){$u$}
	\qbezier(2,3.2)(4,4)(6,3.2)
	\put(4,3.6){\vector(1,0){0.2}}
	\put(3.8,3.9){$ \psi\phi^{-1} $}
	\qbezier(4,0.2)(4,2)(6,3.2)
	\put(4.3,1.6){\vector(1,1){0.2}}
	\put(4.5,1.6){$ \psi $}
	\qbezier(4,0.2)(2,2)(2,3.2)
	\put(2.6,1.7){\vector(-1,1){0.1}}
	\put(2.2,1.6){$ \phi $}
	%\put(4,1.2){\oval(4,2)[r]}
	\end{picture}
	
	\section{Matrix Representation of Linear Transformation}
	
	Let $ T:V\to V $ and $ \mathfrak{B} =\{u_{1},u_{2},...,u_{n}\} $ be the basis for the set V.\\
	If $f(t)$ is a polinomial in $\mathbb{F}$ and $T$ is represented by a diagonal matrix . 
	$
	\Lambda =
	\begin{bmatrix}
	\lambda_{1} & 0 &... & 0 \\
	0 & \lambda_2 & ... & 0 \\
	0 & 0 & ... & 0 \\
	0 & 0 & ... &  \lambda_n  \\
	\end{bmatrix}
	$
	\\
	then $P(T)$ is presented with respect to the same basis by 
	$
	P_{\lambda}
	\begin{bmatrix}
	P_{\lambda_{1}} & 0 &... & 0 \\
	0 & P_{\lambda_2} & ... & 0 \\
	0 & 0 & ... & 0 \\
	0 & 0 & ... &  P_{\lambda_n}  \\
	\end{bmatrix}
	$
	When $T \in L(V)$, does there exist an ordered basis for $V$ with respect to $T$ so that $T$ has a diagonal representation?If such a diagonal representation exists,how to find the ordered basis?\\
	$\mathbf{Example:}$
	Let $ T : \mathbb{P}_{2}(\mathbb{R})\to\mathbb{P}_{2}(\mathbb{R}) $ is a linear transformation defined by $ T(1):=3.1+1.t+(-2).t^2$ , $T(t):=2.1+4.t+(-4).t^2$  and $T(t^2):=2.1+1.t+(-1).t^2 $ . Show that $[T]_{\{1,t,t^2\}}=$
	$	
	\begin{bmatrix}
	3&2&2\\
	1&4&1\\
	-2&-4&-1\\
	\end{bmatrix}	
	$.Does this diagonizable?\\
	$\mathbf{Solution:}$$ T $ is diagonizable if $ \exists $ a base for $ \mathbb{P}_{2}(\mathbb{R}) $	consisting of eigen vectors of $ T $.\\
	$[ T ]- \lambda I =\begin{bmatrix}
    3-\lambda&2&2\\
    1&4-\lambda&1\\
    -2&-4&-1-\lambda\\ 
    \end{bmatrix}
	;\lambda$ is an eigen value of$ [T] $ iff $ [T]-\lambda I $ is singular.$$ det([T]-\lambda I)=0 $$
	The characteristic polynomial  is $$ det([T]-\lambda I)=-(\lambda-1)(\lambda-2)(\lambda-3)$$
	Hence eigen values of $ [T] $ are 1,2 and 3. So algebraic degree is 3. We need to find the eigen vectors corresponding to each eigen value.\\
	When $ \lambda=1 $ , $ [T]-1\lambda=
		\begin{bmatrix}
	2&2&2\\
	1&3&1\\
	-2&-4&-2\\
	\end{bmatrix}
	 $\\-+
	 Rank  ($[T]-1\lambda  $) = 3. All rows are linearly independent.
%	$$ 2x_{1}+2x_{2}+2x_{3}=0$$
	$$ x_{1}+x_{2}+x_{3}=0 $$
	$$ x_{1}+3x_{2}+x_{3}=0 $$
	$$\implies x_{2}=0 , x_{1}=-x_{3}$$
	So the eigen vector corresponding to each eigen value 1 is (1 0 -1).
	$$ P_{1}(t)=1+(-1)t^2 $$
	$ T(P_{1})=T(1+(-1)t^2) \\         
	\hspace*{1.5cm}=T(1)-T(t^2)\\
	\hspace*{1.5cm}=3.1+1.t+(-2).t^2-(2.1+1.t+(-1).t^2)\\
	\hspace*{1.5cm}=1+(-1)t^2 \\
	\hspace*{1.5cm}=1.P_{1}\\
	$
	When $ \lambda=2 $ , $ [T]-2\lambda=
	\begin{bmatrix}
	1&2&2\\
	1&2&1\\
	-2&-4&-3\\
	\end{bmatrix}
	$\\
	Rank  ($[T]-1\lambda  $) = 3. All rows are linearly independent.
	%	$$ 2x_{1}+2x_{2}+2x_{3}=0$$
	$$ x_{1}+2x_{2}+2x_{3}=0 $$
	$$ x_{1}+2x_{2}+x_{3}=0 $$
	$$\implies x_{3}=0 , x_{1}=-2x_{2}$$
	So the eigen vector corresponding to each eigen value 2 is (-2 1 0).
	$$ P_{2}(t)=(-2).1+1.t $$
	$ T(P_{2})=T((-2).1+1.t) \\         
	\hspace*{1.5cm}=(-2)T(1)+T(t)\\
	\hspace*{1.5cm}=-2(3.1+1.t+(-2).t^2)+2.1+4.t+(-4).t^2\\
	\hspace*{1.5cm}=-4+2t\\
	\hspace*{1.5cm}=2.P_{2}\\
	$
	When $ \lambda=3 $ , $ [T]-3\lambda=
	\begin{bmatrix}
	0&2&2\\
	1&1&1\\
	-2&-4&-4\\
	\end{bmatrix}
	$\\
	Rank  ($[T]-1\lambda  $) = 3. All rows are linearly independent.
	%	$$ 2x_{1}+2x_{2}+2x_{3}=0$$
	$$ 2x_{2}+2x_{3}=0 $$
	$$ x_{1}+x_{2}+x_{3}=0 $$
	$$\implies x_{1}=0 , x_{2}=-x_{3}$$
	So the eigen vector corresponding to each eigen value 2 is (0 1 -1).
	$$ P_{3}(t)=1.t+(-1).t^2 $$
	$ T(P_{3})=T(1.t+(-1).t^2) \\         
	\hspace*{1.5cm}=T(t)-T(t^2)\\
	\hspace*{1.5cm}=2.1+4.t+(-4).t^2-(2.1+1.t+(-1).t^2)\\
	\hspace*{1.5cm}=3t-3t^2\\
	\hspace*{1.5cm}=3.P_{3}\\
	$
	For every eigen values $ \exists  $ an eigen vector which are the basis for the new transformation. Since we could find three basis the matrix is diagonizable.\\
	$\ T_{P_{1}}=P_{1} $,
	$ T_{P_{2}}=2P_{2} $ and 
	$ T_{P_{3}}=3P_{3} $\\
	It also can represent as
	$$ T_{P_{1}}=1.P_{1}+0.P_{2}+0.P_{3} $$
	$$ T_{P_{1}}=0.P_{1}+2.P_{2}+0.P_{3} $$
	$$ T_{P_{1}}=0.P_{1}+0.P_{2}+3.P_{3} $$
	So the diagonal matrix equilent to the given linear transformations is 
	$$ [T]_{P_{1},P_{2},P_{3}}=\begin{bmatrix}
	1&0&0\\
	0&2&0\\
	0&0&3\\
	\end{bmatrix} $$
	
	\begin{center}
		\begin{picture}(7,7)
		%\put(0,0){0,0}\put(8,0){8,0}\put(0,8){0,8}\put(8,8){8,8}
		\put(2,2){\circle{4}}\put(6,2){\circle{4}}\put(2,6){\circle{4}}\put(6,6){\circle{4}}
		\qbezier(2,6)(4,7)(6,6)\put(4,6.5){\vector(1,0){0.2}}
		\qbezier(2,2)(4,1.4)(6,2)\put(4,1.689){\vector(1,0){0.2}}
		\qbezier(2,6)(1,4)(2,2)\put(1.5,4){\vector(0,1){0.2}}
		\qbezier(6,6)(5,4)(6,2)\put(5.5,4){\vector(0,1){0.2}}
		\put(4,6.7){$ \Lambda $}\put(1.1,4){$ \phi $}\put(5.1,4){$ \phi $}\put(4,1.3){$ T$}
		\put(1.6,1.8){$ u_{i} $}\put(2.1,2.1){$ \mathbb{V} $}
		\put(5.8,1.6){$Tu_{i} $}\put(6.1,2.1){$ \mathbb{V} $}
		\put(1.6,5.9){$ e_{i} $}\put(1.9,6.3){$ \mathbb{F}^{n} $}
		\put(6,5.9){$ \lambda_{i} e_{i} $}\put(5.65,6.3){$ \mathbb{F}^{n} $}
		\end{picture}
	\end{center}
	\begin{center}
		\begin{picture}(8,10)
		%\put(0,0){0,0}\put(8,0){8,0}\put(0,10){0,10}\put(8,10){8,10}
		\put(2,1){\circle{4}}\put(2,5){\circle{4}}\put(2,9){\circle{4}}\put(6,1){\circle{4}}\put(6,5){\circle{4}}\put(6,9){\circle{4}}
		\qbezier(2,1)(4,2)(6,1)\put(4,1.5){\vector(1,0){0.2}}\put(3.7,1.7){$[T]_{\psi}$}\qbezier(2,5)(4,6)(6,5)\put(4,5.5){\vector(1,0){0.2}}\put(3.9,5.7){$T$}\qbezier(2,9)(4,10)(6,9)\put(4,9.48){\vector(1,0){0.2}}\put(3.7,9.7){$ [T]_{\phi} $}
		\qbezier(2,1)(1.5,3)(2,5)\put(1.75,3){\vector(0,-1){0.2}}\put(1.85,2.85){$\psi$}\qbezier(6,1)(5.5,3)(6,5)\put(5.75,3){\vector(0,-1){0.2}}\put(5.85,2.85){$\psi$}\qbezier(2,9)(1.5,7)(2,5)\put(1.75,7){\vector(0,1){0.2}}\put(1.85,7){$\phi$}\qbezier(6,9)(5.5,7)(6,5)\put(5.75,7){\vector(0,1){0.2}}\put(5.85,7){$\phi$}\qbezier(2,9)(-1,5)(2,1)\put(0.5,5){\vector(0,-1){0.2}}\put(-1.5,4.85){$\psi\phi^{-1}=C$}\qbezier(6,9)(9.,5)(6,1)\put(7.5,5){\vector(0,-1){0.2}}\put(7.67,4.85){$\psi\phi^{-1}=C$}
		\put(1.7,0.6){$\mathbb{F}^{n}$}\put(5.8,0.6){$\mathbb{F}^{n}$}\put(1.7,9.1){$\mathbb{F}^{n}$}\put(6,9.1){$\mathbb{F}^{n}$}\put(6.1,4.9){$\mathbb{V}$}\put(1.6,4.9){$\mathbb{V}$}
		\end{picture}
	\end{center}
	\begin{center}

		\begin{picture}(10,6)
	\put(2,3.2){\circle{2.5}}
	\put(1.5,4){$ \mathbb{F}^n $}
	\put(6,3.2){\circle{2.5}}
	\put(6,4){$ \mathbb{F}^n $}
	\put(4,0.2){\circle{2.5}}
	\put(2.2,0){$ \mathbb{P}_2(\mathbb{R}) $}
	\put(1.3,3.2){$\phi(u)$}
	\put(5.7,3.4){$\psi(u)$}
	\put(4.,-0.1){$u$}
	\qbezier(2,3.2)(4,4)(6,3.2)
	\put(4,3.6){\vector(1,0){0.2}}
	\put(3.8,3.9){$ \psi\phi^{-1} $}
	\qbezier(4,0.2)(4,2)(6,3.2)
	\put(4.3,1.6){\vector(1,1){0.2}}
	\put(4.5,1.6){$ \psi $}
	\qbezier(4,0.2)(2,2)(2,3.2)
	\put(2.6,1.7){\vector(-1,1){0.1}}
	\put(2.2,1.6){$ \phi $}
	%\put(4,1.2){\oval(4,2)[r]}
	\end{picture}
	\end{center}
	
	
\end{document}